\documentclass[12pt,a4paper]{article}
\usepackage[utf8]{inputenc}
\usepackage{amsmath}
\usepackage{breqn}
\usepackage{amsfonts}
\usepackage{amssymb}
\usepackage{graphicx}
\usepackage[margin=0.8in]{geometry}


\begin{document}
\title{\vspace{70mm}\Huge Experimento 02 - Pêndulo de Torção}
\author{ Giovani Garuffi\qquad\hfill
		\textit {RA: 155559}\protect\\
		João Baraldi\hfill
		\textit{RA: 158044}\protect\\
		Lauro Cruz\hfill
		\textit{RA: 156175}\protect\\
		Lucas Schanner\hfill
		\textit{RA: 156412}\protect\\
		Pedro Stringhini\hfill
		\textit {RA: 156983}								
		}
\maketitle
\newpage
\section{Resumo}


\section{Objetivos}


\section{Procedimento Experimental e Coleta de Dados}
\subsection{Materiais utilizados}
\begin{itemize}
	\item Pêndulo de torção com fio metálico
	\item Régua de 1 m
	\item Paquímetro
	\item Micrômetro
	\item Photo-gate
	\item Cronômetro inteligente
\end{itemize}
\subsection{Procedimento}



\subsection{Dados Obtidos}




\section{Análise dos Resultados e Discussões}


\section{Conclusões}


\end{document}
